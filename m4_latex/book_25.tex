\documentclass[fleqn]{book}

%% Language and font encodings 
\usepackage[english]{babel} 
\usepackage[utf8x]{inputenc} 
\usepackage[T1]{fontenc} 
\usepackage{fancyhdr} 
%% Sets page size and margins 
\usepackage[a4paper,top=2cm,bottom=2cm,left=3cm,right=2cm,marginparwidth=1cm]{geometry} 
\usepackage{xcolor} 
%% Useful packages 
\usepackage{amsmath} 
\usepackage{graphicx} 
\usepackage[colorinlistoftodos]{todonotes} 
\usepackage[colorlinks=true, allcolors=blue]{hyperref} 
\usepackage{setspace} 
\setcounter{chapter}{5}
\setcounter{section}{4}
\setlength{\headheight}{14pt}
\newcounter{pro1}
\setcounter{pro1}{30}
\newcommand{\pro}{\par\addtocounter{pro1}{1}
\textbf{Problem \arabic{chapter}.\arabic{pro1} }\quad}
\setcounter{equation}{37}
\definecolor{light-gray}{rgb}{0.8,0.8,0.8}
\begin{document} 
\pagestyle{fancy} 

% этим мы убеждаемся, что заголовки глав и 
% разделов используют нижний регистр. 
\renewcommand{\headrulewidth}{0pt}
\fancyhf{} % убираем текущие установки для колонтитулов 

\fancyhead[RO]{\large \textsl {\textbf{93}}} 
\fancyhead[LO]{\large \textsl{5.4 Successive approximation: How deep is the well?}} 
\Large
\begin{equation} 
\Large t_{1}=T-\frac{h_{0}}{c_{s}}\approx 3.76s.
\end{equation} 
\Large \textrm{In that time, the rock falls a distanc $gt^{2}/2$, so the next approximation to the depth is} 
\begin{equation}
\Large h_{1}=\frac{1}{2}gt_{1}^{2}\approx 70.87m.
\end{equation} 
 \large\textsl{Is this approximate depth an overestimate or underestimate? How accurate is it?} \\ 
\Large \textrm{The calculation of $h_{1}$ used $h_{0}$ to estimate the sound-travel time. Because 
$h_{0}$ overestimates the depth, the procedure overestimates the sound-travel 
time and, by the same amount, underestimates the free-fall time. Thus 
$h_{1}$  underestimates the depth. Indeed, $h_{1}$ is slightly smaller than the true 
depth of roughly $71.56 m$—but by only 1.3\% .} \\ 
\\
\Large\textrm{The method of successive approximation has several advantages over solving the quadratic formula exactly. First, it helps us develop a physical 
understanding of the system; we realize, for example, that most of the 
T = 4 s is spent in free fall, so the depth is roughly $gT^{2}/2$. Second, it 
has a pictorial explanation (Problem 5.34). Third, it gives a sufficiently 
accurate answer quickly. If you want to know whether it is safe to jump 
into the well, why calculate the depth to three decimal places?} \\ 
\\
\Large\textrm{Finally, the method can handle small changes in the model. Maybe the 
speed of sound varies with depth, or air resistance becomes important 
(Problem 5.32). Then the brute-force, quadratic-formula method fails. The 
quadratic formula and the even messier cubic and the quartic formulas 
are rare closed-form solutions to complicated equations. Most equations 
have no closed-form solution. Therefore, a small change to a solvable 
model usually produces an intractable model—if we demand an exact 
answer. The method of successive approximation is a robust alternative 
that produces low-entropy, comprehensible solutions .} \\ 
\\
\colorbox{light-gray}{
\begin{minipage}{\textwidth}
\large\textrm{ \textbf{ \textbf\pro Parameter-value inaccuracies}\\ 
What is $h_{2}$, the second approximation to the depth? Compare the error in $h_{1}$ 
and $h_{2}$ with the error made by using $g=10ms^{-2}$.\\ 
\large\textbf{\pro Effect of air resistance}\\ 
Roughly what fractional error in the depth is produced by neglecting air resistance (Section 2.4.2)? Compare this error to the error in the first approximation \\ 
$h_{1}$ and in the second approximation $h_{2}$ (Problem 5.31).} \\ 
\end{minipage}}
%****************************************************************************************** 
%****************************************************************************************** 
\newpage 
\pagestyle{fancy} 
% этим мы убеждаемся, что заголовки глав и 
% разделов используют нижний регистр. 
\renewcommand{\headrulewidth}{0pt} 
\fancyhf{} % убираем текущие установки для колонтитулов 

\fancyhead[LE]{\large \textsl {\textbf{94}}} 
\fancyhead[RE]{\large \textsl{5 Taking out the big part}} 
\colorbox{light-gray}{
\begin{minipage}{\textwidth}
\large\textrm{\textbf{\pro Dimensionless form of the well-depth analysis} \\ 
Even the messiest results are cleaner and have lower entropy in dimensionless 
form. The four quantities $h$, $g$, $T$, and $c_{s}$
produce two independent dimensionless 
groups (Section 2.4.1). An intuitively reasonable pair are} \\ 
\\ 
\begin{equation} \bar{h}\equiv \frac{h}{gT^{2}} \qquad and \qquad \bar{T}\equiv \frac{gT}{c_{s}} .\end{equation}  \\ 
\textrm{ 
a. What is a physical interpretation of $\bar{T}$?\\ 
b. With two groups, the general dimensionless form is $\bar{h}$ = f($\bar{T}$). What is $\bar{h}$ in 
the easy case $\bar{T}$ $\rightarrow $ 0? \\ 
c. Rewrite the quadratic-formula solution} \\ 
\begin{equation}h=\left ( \frac{-2\sqrt{g}+\sqrt{2/g+4T/c_{s}}}{2/c_{s}} \right )^{2} \end{equation} \\ 
\textrm{as $\bar{h}$= f($\bar{T}$). Then check that f($\bar{T}$) behaves correctly in the easy case $\bar{T}$ $\rightarrow $ 0} \\ 
\\ 
\large\textrm{\textbf{\pro Spacetime diagram of the well depth} \\ 
How does the spacetime diagram $[44]$ illustrate 
the successive approximation of the well depth? 
On the diagram, mark $h_{0}$ (the zeroth approximation to the depth), $h_{1}$ , and the exact depth 
$h$. Mark $t_{0}$ , the zeroth approximation to the 
free-fall time. Why are portions of the rock and 
sound-wavefront curves dotted? How would 
you redraw the diagram if the speed of sound 
doubled? If $g$ doubled?}
\end{minipage}}
\\
\\
\section{ \textbf{Daunting trigonometric integral}}
\Large\textrm{ 
The final example of taking out the big part is to estimate a daunting 
trigonometric integral that I learned as an undergraduate. My classmates 
and I spent many late nights in the physics library solving homework 
problems; the graduate students, doing the same for their courses, would 
regale us with their favorite mathematics and physics problems. 
The integral appeared on the mathematical-preliminaries exam to enter 
the Landau Institute for Theoretical Physics in the former USSR. The 
problem is to evaluate}\\ 
\\ \begin{equation}
\int_{-\pi /2}^{\pi /2}(\cos t)^{100}dt
\end{equation} 
%************************************************************************************* 
%*********************************************************************************** 
\newpage 
\pagestyle{fancy} 
% этим мы убеждаемся, что заголовки глав и 
% разделов используют нижний регистр. 
\renewcommand{\headrulewidth}{0pt} 
\fancyhf{} % убираем текущие установки для колонтитулов 

\fancyhead[RO]{\large \textsl {\textbf{95}}} 
\fancyhead[LO]{\large \textsl{5.5 Daunting trigonometric integral}} 
\Large\textrm{to within 5 \% in less than 5 min without using a calculator or computer!\\ 
That $(\cos t)^{100}$ looks frightening. Most trigonometric identities do not 
help. The usually helpful identity $(\cos t)^{2}=(\cos 2t -1)/2$ produces only} 
\begin{equation}
(\cos t)^{100}=\left ( \frac{\cos 2t-1}{2} \right )^{50} , 
\end{equation}

\Large \textrm{which becomes a trigonometric monster upon expanding the $50th$ power. 
A clue pointing to a simpler method is that 5\% accuracy is sufficient—so, 
find the big part! The integrand is largest when $t$ is near zero. There,
$\cot t \approx 1-t^{2}/2$ (Problem 5.20), so the integrand is roughly}
\begin{equation}
(\cos t)^{100}\approx \left (1- \frac{t^{2}}{2} \right )^{100} . 
\end{equation}
\Large\textrm{It has the familiar form $(1 + z)^{n}$, with fractional change $z = −t^{2}/2$ and 
exponent $n = 100$. When $t$ is small, $z = −t^{2}/2$ is tiny, so $(1 + z)^{n}$ may be 
approximated using the results of Section 5.3.4:}
\begin{flushleft}
\begin{equation}
 (1+z)^{n}\approx \left\{\begin{matrix} 1+nz &  (z\ll 1) \quad \textrm{and} \quad nz\ll 1 &\\ e ^{nz}& (z\ll 1 \quad \textrm{and} \quad nz \quad \textrm{unrestricted}) .& \end{matrix}\right.  
\end{equation}
\end{flushleft}
\Large \textrm{Because the exponent $n$ is large, $nz$ can be large even when $t$ and $z$ are 
small. Therefore, the safest approximation is $(1 + z)^{n}\approx e ^{nz}$; then} 
\begin{equation}
(\cos t)^{100}\approx \left (1- \frac{t^{2}}{2} \right )^{100} \approx e ^{-50t^{2}} .
\end{equation}
\Large\textrm{A cosine raised to a high power becomes a Gaussian! 
As a check on this surprising conclusion, computergenerated plots of $(\cos t)^{n}$ for $n = 1...5$ show a 
Gaussian bell shape taking form as $n$ increases. \\
Even with this graphical evidence, replacing $(\cos t)^{100}$ by a Gaussian is a 
bit suspicious. In the original integral, $t$ ranges from $-\pi /2$ to $\pi /2$ , and 
these endpoints are far outside the region where $\cot t \approx 1-t^{2}/2$ is an 
accurate approximation. Fortunately, this issue contributes only a tiny 
error (Problem 5.35). Ignoring this error turns the original integral into a 
Gaussian integral with finite limits: } 
\begin{equation}
\int_{-\pi /2}^{\pi /2}(\cos t)^{100}dt \approx \int_{-\pi /2}^{\pi /2} e ^{-50t^{2}}dt .
\end{equation}


%************************************************************************************* 
%*********************************************************************************** 
\newpage 
\pagestyle{fancy} 
% этим мы убеждаемся, что заголовки глав и 
% разделов используют нижний регистр. 
\renewcommand{\headrulewidth}{0pt} 
\fancyhf{} % убираем текущие установки для колонтитулов 

\fancyhead[LE]{\large \textsl {\textbf{96}}} 
\fancyhead[RE]{\large \textsl{5 Taking out the big part}} 
\Large\textrm{Unfortunately, with finite limits the integral has no closed form. But 
extending the limits to infinity produces a closed form while contributing 
almost no error (Problem 5.36). The approximation chain is now } 
\begin{equation}
\int_{-\pi /2}^{\pi /2}(\cos t)^{100}dt \approx \int_{-\pi /2}^{\pi /2} e ^{-50t^{2}}dt \approx \int_{-\infty }^{\infty } e ^{-50t^{2}}dt. \end{equation}\\ 
\colorbox{light-gray}{
\begin{minipage}{\textwidth}
\large\textrm{\textbf{\pro Using the original limits} \\ 
The approximation $\cot t \approx 1-t^{2}/2$ requires that $t$ be small. Why doesn’t using 
the approximation outside the small-t range contribute a significant error?} \\ 

\textrm{\textbf{\pro Extending the limits} \\ 
Why doesn’t extending the integration limits from $\pm\pi /2$ to $\pm\infty$ contribute a 
significant error?} \\ 
 \end{minipage}}
\Large\textrm{The last integral is an old friend (Section 2.1):$\int_{-\infty}^{\infty} e^{-\alpha t^{2}} dt =\sqrt{\pi /\alpha }$. With 
$\alpha$ = 50, the integral becomes $\pi$/50. Conveniently, 50 is roughly 16$\pi$, so 
the square root—and our 5\% estimate—is roughly 0.25. 
For comparison, the exact integral is (Problem 5.41) 
} 
\begin{equation}
\int_{-\pi /2}^{\pi /2}(\cos t)^{100}dt=2^{-n}\begin{pmatrix} 
n \\ 
n/2 
\end{pmatrix} \pi .  
\end{equation} 
\Large\textrm{When $n = 100$, the binomial coefficient and power of two produce} \\ 
\begin{equation}
\large \frac{12611418068195524166851562157}{158456325028528675187087900672}\pi \approx 0.25003696348037. \end{equation}  \\ 
\Large\textrm{Our 5-minute, within-5\% estimate of 0.25 is accurate to almost 0.01\%!} \\ 
\colorbox{light-gray}{
\begin{minipage}{\textwidth}

\large\textrm{\textbf { \pro Sketching the approximations} \\  
Plot $(\cos t)^{100}$ and its two approximations $ e ^{-50t^{2}}$ 
and 1 $-50t^{2}$.} \\
\large\textrm{\textbf{\pro Simplest approximation} \\  
Use the linear fractional-change approximation $(1 − t^{2}/2)^{100} \approx 1 − 50t^{2}$ to 
approximate the integrand; then integrate it over the range where $1 − 50t^{2}$ is 
positive. How close is the result of this 1-minute method to the exact value 
$0.2500 . . .$?} \\ 
\\ 
\large\textrm{\textbf{\pro Huge exponent} \\ 
\\ 
Estimate} 
\\ 
\begin{equation} 
\int_{-\pi /2}^{\pi /2}(\cos t)^{100}dt .
 \end{equation}
 \end{minipage}}
\end{document}
